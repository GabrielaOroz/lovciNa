
\chapter{Zaključak i budući rad}
		
		
		 Naš je zadatak bio napraviti aplikaciju koja služi za praćenje divljih životinja te olakšava organizaciju posla istraživačima, voditeljima postaje te tragačima koji sudjeljuju u takvim akcijama.
		 
		 Aplikaciju smo radili jedan semestar te smo stekli iskustvo rada u timu koje je vrlo korisno za našu buduću karijeru. Koliko god smo mislili da znamo funkcionirati i raditi u timu, ovaj projekt nam je pokazao da je zapravo vrlo izazovno raditi u timu od sedam osoba te u određenom roku napraviti proizvod koji ima sve željene funkcionalnosti. Osim toga, većini tima ova je aplikacija prvi veliki projekt pa smo bogatiji i za jedno iskustvo izrade takve aplikacije.
		 
		 Tijekom razvoja aplikacije, proces koji je trajao najduže bila je izrada i implementacija baze podataka. Bilo je izazovno osmisliti bazu koja će dobro funkcionirati u našem zadatku s obzirom da se prvi put susrećemo s tako kompleksnom bazom te smatramo da je to vještina koja nam je možda nedostajala, a znatno bi ubrzala naš rad. 
		 
		 Osim toga, šest od sedam članova našeg tima nikada nije radilo s tehnologijama i alatima koje smo koristili pri izradi ovog projekta te smo tako svi morali vrlo ubrzano učiti i međusobno si pomagati. Tako smo zbog manjka iskustva u više situacija morali izbrisati i ponovno pisati kod za neku funkcionalnost kada bismo shvatili da je ono što smo radili netočno.
		 
		 Bili smo podijeljeni u dva podtima, jedan za frontend te jedan za backend. Svaki podtim imao je voditelja koji je kontrolirao što ostali članovi rade te smatramo da je to vrlo dobro funkcioniralo. 
		 
		 Implementirali smo sve funkcionalnosti koje su zatražene u zadatku te smo, iako naravno ima mjesta za napredak na implementaciji te dodatnim funkcionalnostima aplikacije, vrlo zadovoljni izrađenom aplikacijom s obzirom na sve nabrojane čimbenike.
		 
		 
		 
		
		
		\eject 