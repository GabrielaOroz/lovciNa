\chapter{Specifikacija programske potpore}
		
	\section{Funkcionalni zahtjevi}
			
			\textbf{\textit{dio 1. revizije}}\\
			
			\textit{Navesti \textbf{dionike} koji imaju \textbf{interes u ovom sustavu} ili  \textbf{su nositelji odgovornosti}. To su prije svega korisnici, ali i administratori sustava, naručitelji, razvojni tim.}\\
				
			\textit{Navesti \textbf{aktore} koji izravno \textbf{koriste} ili \textbf{komuniciraju sa sustavom}. Oni mogu imati inicijatorsku ulogu, tj. započinju određene procese u sustavu ili samo sudioničku ulogu, tj. obavljaju određeni posao. Za svakog aktora navesti funkcionalne zahtjeve koji se na njega odnose.}\\
			
			
			\noindent \textbf{Dionici:}
			
			\begin{packed_enum}
				
				\item Dionik 1
				\item Dionik 2				
				\item ...
				
			\end{packed_enum}
			
			\noindent \textbf{Aktori i njihovi funkcionalni zahtjevi:}
			
			
			\begin{packed_enum}
				\item  \underbar{Aktor 1 (inicijator) može:}
				
				\begin{packed_enum}
					
					\item funkcionalnost 1
					\item funkcionalnost 2
					\begin{packed_enum}
						
						\item  podfunkcionalnost 1 
						\item  podfunkcionalnost 2
				
					\end{packed_enum}
					\item  funkcionalnost 3
					
				\end{packed_enum}
			
				\item  \underbar{Aktor 2 (sudionik) može:}
				
				\begin{packed_enum}
					
					\item funkcionalnost 1
					\item funkcionalnost 2
					
				\end{packed_enum}
			\end{packed_enum}
			
			\eject 
			
			
				
			\subsection{Obrasci uporabe}
				
				\textbf{\textit{dio 1. revizije}}
				
				\subsubsection{Opis obrazaca uporabe}
					\textit{Funkcionalne zahtjeve razraditi u obliku obrazaca uporabe. Svaki obrazac je potrebno razraditi prema donjem predlošku. Ukoliko u nekom koraku može doći do odstupanja, potrebno je to odstupanje opisati i po mogućnosti ponuditi rješenje kojim bi se tijek obrasca vratio na osnovni tijek.}\\
					
					\noindent \underbar{\textbf{UC$<$broj obrasca$>$ -$<$Stvaranje zadataka$>$}}
					\begin{packed_item}
						
						\item \textbf{Glavni sudionik: }$<$Istraživač$>$
						\item  \textbf{Cilj:} $<$Stvaranje zadataka$>$
						\item  \textbf{Sudionici:} $<$Baza podataka$>$
						\item  \textbf{Preduvjet:} $<$Istraživač je prijavljen$>$
						\item  \textbf{Opis osnovnog tijeka:}
						
						\item[] \begin{packed_enum}
							
							\item $<$Istraživaču je omogućen odabir dodavanja zadataka pojedinom tragaču.$>$
							\item $<$Istraživač odabire zadatak (prolazak određenom rutom i dolazak do određene lokacije, postavljanje
							kamere ili uređaja za praćenje) i dodjeljuje ga tragaču.$>$
							\item $<$Tragačima je omogućen pregled zadataka za određenu akciju.$>$
							\item $<$Dodijeljeni zadatci se spremaju u bazu podataka.$>$
						\end{packed_enum}
						
					\end{packed_item}
					
					\noindent \underbar{\textbf{UC$<$broj obrasca$>$ -$<$Micanje tragača s akcije$>$}}
					\begin{packed_item}
						
						\item \textbf{Glavni sudionik: }$<$Tragač$>$
						\item  \textbf{Cilj:} $<$Micanje tragača s akcije$>$
						\item  \textbf{Sudionici:} $<$Baza podataka$>$
						\item  \textbf{Preduvjet:} $<$Tragač je prijavljen$>$
						\item  \textbf{Opis osnovnog tijeka:}
						
						\item[] \begin{packed_enum}
							
							\item $<$Tragaču je omogućeno micanje s akcije.$>$
							\item $<$Tragač se miče s akcije prilikom završetka svih zadataka i potvrđuje micanje.$>$
							\item $<$Akcija određenog tragača se briše iz baze podataka.$>$
						\end{packed_enum}
						
					\end{packed_item}
					
					\noindent \underbar{\textbf{UC$<$broj obrasca$>$ -$<$Prikaz toplinskih karata za analizu kretanja životinja$>$}}
					\begin{packed_item}
						
						\item \textbf{Glavni sudionik: }$<$Istraživač$>$
						\item  \textbf{Cilj:} $<$Prikaz toplinskih karata za analizu kretanja životinja$>$
						\item  \textbf{Sudionici:} $<$Baza podataka$>$
						\item  \textbf{Preduvjet:} $<$Istraživač je prijavljen$>$
						\item  \textbf{Opis osnovnog tijeka:}
						
						\item[] \begin{packed_enum}
							
							\item $<$Istraživač odabire opciju prikaza toplinskih karata za analizu kretanja životinja.$>$
							\item $<$Sustav omogućuje odabir između različitih podataka (povijesne pozicije životinja i trenutne pozicije životinja) koji se mogu filtrirati (po vrsti ili pojedinačno po jedinki).$>$
							\item $<$Istraživač odabire vrstu podatka koju želi prikazati na karti.$>$
							\item $<$Sustav generira toplinsku kartu na temelju odabranih podataka.$>$
						\end{packed_enum}
						
					\end{packed_item}
					
					\noindent \underbar{\textbf{UC$<$broj obrasca$>$ -$<$Prikaz pozicija tragača na nekoj akciji na karti$>$}}
					\begin{packed_item}
						
						\item \textbf{Glavni sudionik: }$<$Korisnik$>$
						\item  \textbf{Cilj:} $<$Prikaz pozicija tragača na nekoj akciji na karti$>$
						\item  \textbf{Sudionici:} $<$Baza podataka$>$
						\item  \textbf{Preduvjet:} $<$Korisnik je prijavljen$>$
						\item  \textbf{Opis osnovnog tijeka:}
						
						\item[] \begin{packed_enum}
							
							\item $<$Korisnik odabire opciju prikaza pozicija tragača na karti.$>$
							\item $<$Sustav omogućuje odabir između različitih podataka (povijesne pozicije tragača i trenutne pozicije tragača) koji se mogu filtrirati (po tipu prijevoza ili pojedinačno po tragaču).$>$
							\item $<$Korisnik odabire vrstu podatka koju želi prikazati na karti.$>$
							\item $<$Sustav generira toplinsku kartu na temelju odabranih podataka.$>$
						\end{packed_enum}
						
					\end{packed_item}
					
					\noindent \underbar{\textbf{UC$<$broj obrasca$>$ -$<$Ostavljanje komentara na karti za određeni zadatak$>$}}
					\begin{packed_item}
						
						\item \textbf{Glavni sudionik: }$<$Istraživač$>$
						\item  \textbf{Cilj:} $<$Ostavljanje komentara na karti za određeni zadatak$>$
						\item  \textbf{Sudionici:} $<$Baza podataka$>$
						\item  \textbf{Preduvjet:} $<$Istraživač je prijavljen$>$
						\item  \textbf{Opis osnovnog tijeka:}
						
						\item[] \begin{packed_enum}
							
							\item $<$Istraživač odabire zadatak na koji želi ostaviti komentar.$>$
							\item $<$Istraživač unosi komentar i potvrđuje unos.$>$
							\item $<$Komentar se pohranjuje u bazu podataka.$>$
							\item $<$Sudionici akcije mogu vidjeti unešeni komentar.$>$
						\end{packed_enum}
						
					\end{packed_item}
					
					\noindent \underbar{\textbf{UC$<$broj obrasca$>$ -$<$Ostavljanje komentara na karti za određenu životinju unutar neke akcije$>$}}
					\begin{packed_item}
						
						\item \textbf{Glavni sudionik: }$<$Tragač$>$
						\item  \textbf{Cilj:} $<$Ostavljanje komentara na karti za određenu životinju unutar neke akcije$>$
						\item  \textbf{Sudionici:} $<$Baza podataka$>$
						\item  \textbf{Preduvjet:} $<$Tragač je prijavljen$>$
						\item  \textbf{Opis osnovnog tijeka:}
						
						\item[] \begin{packed_enum}
							
							\item $<$Tragač odabire životinju na koju želi ostaviti komentar.$>$
							\item $<$Tragač unosi komentar i potvrđuje unos.$>$
							\item $<$Komentar se pohranjuje u bazu podataka.$>$
							\item $<$Sudionici akcije mogu vidjeti unešeni komentar.$>$
						\end{packed_enum}
						
					\end{packed_item}
				
					
				\subsubsection{Dijagrami obrazaca uporabe}
					
					\textit{Prikazati odnos aktora i obrazaca uporabe odgovarajućim UML dijagramom. Nije nužno nacrtati sve na jednom dijagramu. Modelirati po razinama apstrakcije i skupovima srodnih funkcionalnosti.}
				\eject		
				
			\subsection{Sekvencijski dijagrami}
				
				\textbf{\textit{dio 1. revizije}}\\
				
				\textit{Nacrtati sekvencijske dijagrame koji modeliraju najvažnije dijelove sustava (max. 4 dijagrama). Ukoliko postoji nedoumica oko odabira, razjasniti s asistentom. Uz svaki dijagram napisati detaljni opis dijagrama.}
				\eject
	
		\section{Ostali zahtjevi}
		
			\textbf{\textit{dio 1. revizije}}\\
		 
			 \textit{Nefunkcionalni zahtjevi i zahtjevi domene primjene dopunjuju funkcionalne zahtjeve. Oni opisuju \textbf{kako se sustav treba ponašati} i koja \textbf{ograničenja} treba poštivati (performanse, korisničko iskustvo, pouzdanost, standardi kvalitete, sigurnost...). Primjeri takvih zahtjeva u Vašem projektu mogu biti: podržani jezici korisničkog sučelja, vrijeme odziva, najveći mogući podržani broj korisnika, podržane web/mobilne platforme, razina zaštite (protokoli komunikacije, kriptiranje...)... Svaki takav zahtjev potrebno je navesti u jednoj ili dvije rečenice.}
			 
			 
			 
	